% Licensed under GPL, restrictions are only imposed to the .tex file and its edits. There are no restrictions over the output of this program.
%Author: David Martínez Rubio

\documentclass[11pt]{article}

% Estilo del documento
\usepackage[utf8]{inputenc} 		% Permite escribir acentos normalmente
\usepackage[T1]{fontenc}  			% Permite caracterse UTF-8 en el codigo
\usepackage{geometry}				% Permite editar los márgenes del documento y su formato
\usepackage[english]{babel} 		% Tipo de letra e idioma
\usepackage{indentfirst}			% Tabula el primer párrafo de un cada seccion/subseccion
\usepackage{hyperref}				% Referencias dentro del documento e hipervínculos fuera
\usepackage{marvosym} 				% pintar un teléfono movil y una carta \Mobilefone
\usepackage{multirow}				% Agrupar filas en tablas
\usepackage{xcolor}					% Carga algunos colores por defecto y permite crearlos
\usepackage{titlesec}				% Para cambiar el estilo de la sección

% Estilo del documento
\definecolor{mygreen}{rgb}{0,0.6,0}
\titleformat*{\section}{\Large\bfseries\color{mygreen}}	%Cambiamos el color del título de sección
\geometry{margin=2cm}					% Margen de 3cm
\geometry{a4paper}						% Papel A4
\setlength{\parskip}{0.5\baselineskip}	% Separación entre párrafos
\renewcommand{\arraystretch}{1.5}		% Esto es para cambiar el espacio entre filas dentro del entorno tabular
\hypersetup{colorlinks=true, hidelinks}	% Los enlaces son clickables pero no recuadrados
\pagenumbering{gobble}					%deshabilitamos números de página

% Paragraph Molon
\makeatletter
\renewcommand{\paragraph}{\@startsection{paragraph}{4}{0ex}%
   {-3.25ex plus -1ex minus -0.2ex}%
   {1ex plus 0.2ex}%
   {\normalfont\normalsize\bfseries}}
\makeatother

%comienzo del documento
\begin{document}

\begin{tabular}{l r}

\multirow{3}{*}{\Huge \textbf{Nombre Apellido1 Apellido2}} 

										& \small Calle falsa 123 \\
										& \small Cuidad (Pais) Postal Code: 12345\\
										& \small \Mobilefone \ \ (+34) 666666666 \\
\huge \textcolor{gray}{\textbf{Curriculum Vitae}}					&\small \Letter \ \ \href{mailto:hola@gmail.com}{hola@gmail.com}

\end{tabular}


\section*{Educación}
	
\begin{tabular}{p{2cm} p{13.5cm}}	
Septiembre $2012$-Ahora		& \textbf{Mi carrera en mi universidad (país). } 
\\
							& \textbf{Temario:}
							
							\url{http://www.enlace_al_temario.com} Se puede poner una breve descripción
							
							\textbf{Fecha de terminacion esperada:} Junio 2013 
							
							\textbf{Nota media:} 10/10
\\

Agosto, Septiembre $2014$ 	& \textbf{Curso de noseque que hice en verano}
\\
							& Descripción del curso \url{https://www.url/al/curso}
\\

Hasta Junio de $2012$ 			& \textbf{Bachillerato de tal tipo} 
\\

							& Mi instituto, ciudad, provincia país.
\\
\end{tabular}

\section*{Premios y reconocimientos} %Varias entradas como esta
\begin{tabular}{r p{14cm}}
$2015$ & \textbf{$\textbf{17}^{o}$ puesto} Hackaton internacional en San Benito de la Valduerna, país
\\

$2015$ &\textbf{Medalla de bronce}, Olimpiada de informática
\\
\end{tabular}

\section*{Idiomas}

\begin{tabular}{r l}
Español & \textbf{Lengua materna} \\
Inglés & \textbf{Fluido} \\
Francés & \textbf{Básico}

\end{tabular}

\section*{Proyectos}
\begin{tabular}{r p{14cm}}
$2014$ & Descripción del proyecto.
\\
\end{tabular}

\section*{Experiencia laboral}
\begin{tabular}{r p{14cm}}
$2014$ & Descripción.
\\
\end{tabular}

\section*{Computer skills}

\begin{tabular}{r l}
\textbf{Sistemas operativos} & GNU/Linux , Windows, Android \\

\textbf{Lenguajes de programación} 	& C++, C, Java, Haskell\\
								& SQL (Oracle)\\
								& MATLAB, Octave\\
								& VHDL, ARM Assembly\\
								& programación Android, Python, Prolog, Bash \\

\textbf{Redacción}			& LaTeX, Libre Office, Microsoft Office\\

\textbf{IDE}					& Vim, Eclipse, Netbeans, Microsoft Visual Studio, Xilinx \\


\textbf{Cuenta de github}				& \url{https://github.com/eticaasdf}\\
\end{tabular}


\end{document}
